\section{Lost \& Found: Predicting Locations from
Images}\label{lost-found-predicting-locations-from-images}

Team name: ExampleTeam

\textbf{Group Members} \href{https://gitlab.com/Killusions}{Linus
Schlumberger} \href{https://gitlab.com/Valairaa}{Lukas Stöckli}
\href{https://gitlab.com/yusigrist}{Yutaro Sigrist}

\begin{Shaded}
\begin{Highlighting}[]
\NormalTok{title: \# Table of content}
\NormalTok{style: nestedList}
\NormalTok{includeLinks: true}
\end{Highlighting}
\end{Shaded}

\section{Introduction}\label{introduction}

\subsection{Problem description}\label{problem-description}

Nowadays, images are often automatically enriched with various data from
different sensors within devices, including location metadata. However,
this metadata often gets lost when images are sent through multiple
applications or when devices are set not to track locations for privacy
reasons. As a result, images may initially have metadata, but it is lost
when shared with friends or published online. This raises the question:
Is it possible to re-enrich these images with their location after the
metadata is lost?

The main goal of this project is to determine if an Image Classification
Model can outperform humans in guessing the countries or regions of
images based solely on images with low resolution and no additional
information.

\subsubsection{Project Overview}\label{project-overview}

This project explores the development of an Image Classification model,
focusing on simple street-view images grouped by countries to predict
the country where an image was taken. Given limited prior experience
with Image Classification, this initiative aims to enhance understanding
and skills in this domain. The first objective is to create a model
capable of identifying the country from a given image. Building upon
this, a second model will be developed to predict the exact region of
the image, providing a more precise location than just the country.

The main goal is to develop a robust Image Classification model that can
serve as a foundational tool for various applications. This overarching
objective supports the specific sub-goals of predicting the country and
coordinates of an image. This leads to the question: for what main
purposes could an image classifier for countries or coordinates be
valuable? By exploring potential applications, the project aims to
demonstrate the broader utility of the developed models in real-world
scenarios.

\subsubsection{Potential Applications}\label{potential-applications}

\begin{itemize}
\tightlist
\item
  \textbf{Helping find missing persons}: Our solution can help find
  where missing people might be by analyzing pictures shared publicly.
  The emotional impact of helping reunite families or providing
  important clues is huge. Especially when the model will be used in
  addition to the search process for the police. For missing people,
  every second counts after a kidnapping, especially when the search is
  international.
\item
  \textbf{Rediscovering memories and family history}: Have you ever come
  across an old image of someone close to you? Maybe of a deceased
  family member or someone who may just not remember where it was taken.
  Our model can try to predict the rough location to help you rediscover
  your past.
\item
  \textbf{Supporting humanitarian action:} In disaster situations, it
  could help to quickly identify the most affected areas by analyzing
  current images from social media or aid organizations. This would
  improve the coordination of rescue and relief efforts and offer hope
  and support to those impacted.
\item
  \textbf{Discovering new travel destinations:} Have you ever
  encountered stunning images of places on Instagram or other social
  media platforms and wondered where they were taken? Our image
  classifier can help you with that. By analyzing the image, our
  classifier can identify the location and provide you with the
  information you need to plan your next visit to this amazing place.
  This way, you can discover new and exciting travel destinations that
  you may have never known about before.
\item
  \textbf{Classification as a service}: With this service, we will help
  other companies or data science projects label their data. Sometimes
  companies want to block, permit, or deploy individual versions of
  their applications in different countries. Some countries have more
  restrictions for deploying applications, therefore the image predictor
  can help the companies have the right version on the right devices for
  these countries.
\end{itemize}

\subsection{Literature Review}\label{literature-review}

\subsubsection{State of the Art}\label{state-of-the-art}

Recent advancements in deep learning have significantly enhanced the
ability to determine the geographical location of an image. DeepGeo,
developed by Suresh et al., leverages visual cues such as vegetation and
man-made structures like roads and signage to infer locations. This
approach aims to replicate the human ability to use environmental
indicators and prior knowledge for geolocation (Suresh et al., 2018).
The DeepGeo model restricts its scope to the United States, utilizing
panoramic viewpoints to classify images based on their state. Each input
sample consists of four images taken at the same location, oriented in
cardinal directions, which are then classified into one of 50 state
labels (Suresh et al., 2018){[}Suresh, Chodosh, and Abello (2016);{]}.

In contrast, PlaNet, developed by Weyand et al., tackles global image
geolocation. It employs a deep convolutional neural network based on the
Inception architecture, trained on 126 million geotagged photos from
Flickr. PlaNet's method involves partitioning the world map into
multi-scale geographic cells and classifying test images into these
cells. Despite its large dataset and extensive training, PlaNet achieves
a country-level accuracy of only 30\% on its test set (Weyand et al.,
2016). M2GPS, developed by Hays and Efros, is another significant
baseline in scalable image geolocation. This model performs data-driven
localization by computing the closest match via scene matching with a
large corpus of 6 million geotagged Flickr images, utilizing features
such as color and geometric information. IM2GPS's approach demonstrates
the importance of leveraging large datasets for effective geolocation
(Hays and Efros, 2008){[}Hays and Efros
(2008);{]}\hspace{0pt}\hspace{0pt}.

Banerjee's work emphasizes the classification task of predicting image
location solely based on pixel data. Their research highlights the use
of CNNs and transfer learning to achieve high-accuracy models capable of
superhuman performance. CNNs are particularly effective due to their
ability to capture low-level and complex spatial patterns (Banerjee,
2023). Dayton et al.~explored a similar task by using a ResNet-50 CNN
pre-trained on ImageNet for classifying street view images from the game
GeoGuessr. Their model utilized transfer learning to refine the
pre-trained network on a dataset specifically curated for the task,
resizing images to 224x224 pixels for input. By fine-tuning the last
layers of ResNet-50, they achieved a test accuracy of over 70\% for 20
different countries, highlighting the efficacy of leveraging pre-trained
models for geolocation tasks (Dayton et al., 2024) {[}Dayton, Heo, and
Werner (2023);{]}.

Another notable model is PIGEON, which combines semantic geocell
creation with multi-task contrastive pretraining and a novel loss
function. PIGEON is trained on GeoGuessr data and demonstrates the
capability to place over 40\% of its guesses within 25 kilometers of the
target location globally, which is remarkable. This model highlights the
importance of using diverse datasets and innovative training techniques
to enhance geolocation accuracy (Haas et al., 2024). While these models
exhibit high accuracy in controlled conditions, they often rely on
high-resolution images, multiple perspectives, and enriched datasets
that do not reflect real-world scenarios. For instance, DeepGeo's use of
panoramic images and PlaNet's extensive dataset of geotagged Flickr
photos introduce biases towards urban areas and well-known landmarks,
limiting their effectiveness in arbitrary or rural locations (Suresh et
al., 2018). Additionally, these models struggle to generalize to
lower-resolution images and more diverse datasets that include unseen
locations, as highlighted by the performance discrepancies observed in
models like PIGEON when applied to varied datasets (Haas et al., 2024)
{[}Haas et al. (2024);{]}.

Furthermore, Banerjee's research on digital image classification since
the 1970s underscores the evolution from using textural and colour
features to the current reliance on CNNs. This historical perspective
reveals that early models had limited discriminative power and
robustness, which were significantly improved with the advent of SIFT
(Scale-invariant feature transform) and visual Bag-of-Words models.
However, the transition to CNNs marked a pivotal shift due to their
superior ability to capture both low-level and high-level features
(Banerjee, 2023){[}Banerjee (2023);{]}. Dayton et al.~further illustrate
the application of transfer learning in geolocation by refining a
pre-trained ResNet-50 model on a specific geolocation task. Their work
highlights the importance of data augmentation and hyperparameter tuning
in improving model performance, as well as the need for balanced
datasets to reduce bias and enhance generalizability (Dayton et al.,
2024){[}Dayton, Heo, and Werner (2023);{]}.

To develop more robust and universally applicable geolocation models, it
is essential to focus on creating systems that can operate effectively
with lower-resolution images and without the need for panoramic views or
extensive enriched datasets. This involves training models on diverse,
real-world datasets that include a variety of image types, from urban
streets to rural landscapes, captured under different conditions and
perspectives. By doing so, the models can better mimic the conditions
under which humans typically use images for geolocation, such as in
social media posts, emergency situations, or historical photo analysis.
For instance, PIGEOTTO, an evolution of PIGEON, takes a single image per
location and is trained on a larger, highly diverse dataset of over 4
million photos from Flickr and Wikipedia, excluding Street View data.
This approach demonstrates the model's ability to generalize to unseen
places and perform well in realistic scenarios without the need for
multiple images per location (Haas et al., 2024) {[}Haas et al.
(2024);{]}.

\subsubsection{Recent Breakthroughs in CNN
Architectures}\label{recent-breakthroughs-in-cnn-architectures}

\paragraph{ResNet: Addressing the Degradation Problem in Deep
Networks}\label{resnet-addressing-the-degradation-problem-in-deep-networks}

The introduction of deep residual learning by He et al.~(2015) marked a
significant milestone in the development of convolutional neural
networks (CNNs). Their work addressed the degradation problem in deep
neural networks by proposing a residual learning framework that allows
layers to learn residual functions with reference to the layer inputs.
This architecture, known as ResNet, employs shortcut connections that
perform identity mapping, which are then added to the outputs of the
stacked layers (He et al., 2015) {[}He et al. (2015);{]}. This
innovative approach not only mitigates the vanishing gradient problem
but also enables the training of extremely deep networks with more than
100 layers, achieving superior performance in image classification
tasks.

\paragraph{MobileNetV2: Inverted Residuals and Linear
Bottlenecks}\label{mobilenetv2-inverted-residuals-and-linear-bottlenecks}

Designed specifically for mobile and resource-constrained environments,
MobileNetV2 introduced by Sandler et al.~(2019) represents a significant
advancement in efficient CNN architectures. The core innovation of
MobileNetV2 is the inverted residual with linear bottleneck layer
module, which significantly reduces the memory footprint and
computational cost during inference without sacrificing accuracy
(Sandler et al., 2019) {[}Sandler et al. (2019);{]}. This is achieved
through a low-dimensional compressed representation that is expanded and
then filtered with a lightweight depthwise convolution before being
projected back to a low-dimensional representation.

\paragraph{EfficientNet: Rethinking Model Scaling for
CNN's}\label{efficientnet-rethinking-model-scaling-for-cnns}

EfficientNet, proposed by Tan and Le (2020), introduces a new model
scaling method that uniformly scales all dimensions of depth, width, and
resolution using a simple yet highly effective compound scaling method.
This balanced scaling approach enables EfficientNet to achieve superior
performance while being much smaller and faster than previous models
(Tan and Le, 2020) {[}Tan and Le (2020);{]}. For instance,
EfficientNet-B7 achieves state-of-the-art 84.3\% top-1 accuracy on
ImageNet, being 8.4x smaller and 6.1x faster on inference compared to
traditional CNN architectures (Tan and Le, 2020) {[}Tan and Le
(2020);{]}.

\subsection{Contributions}\label{contributions}

This paper has four main contributions. Firstly, we address the
limitations of current geolocation models by developing a novel approach
that leverages low-resolution images, enabling accurate geolocation in
more realistic and diverse scenarios. Secondly, we enhance the dataset
by expanding it to include more countries, ensuring a balanced and
distributed representation, which is crucial for mitigating biases
present in state-of-the-art models. Thirdly, we tackle hardware
limitations by optimizing image sizes, making the model more accessible
and efficient for deployment on various hardware platforms. Finally, we
propose a new methodology for training and fine-tuning our model,
incorporating the latest advancements in transfer learning and data
augmentation techniques, which significantly improve the model's
performance and generalizability across different real-world
applications.

Our contributions aim to advance the field of image geolocation, making
it more practical and effective for a wide range of applications, from
aiding in disaster response to rediscovering family histories and
beyond. By addressing these key challenges, we believe our work will
pave the way for the development of more robust and universally
applicable geolocation technologies.

\section{Methods (REPRODUCIBILIY is the main
goal)}\label{methods-reproducibiliy-is-the-main-goal}

\subsection{Data collection}\label{data-collection}

\subsubsection{Data source}\label{data-source}

When it comes to relatively uniform street imagery, there are not many
sources. Google Street View \textless-LINK\textgreater{} being by far
the biggest. But instead of sourcing our images directly from Google, we
wanted to have a more representative distribution, as well as a more
interactive demonstration.

For this reason we instead opted for the online Geography game called
Geoguessr \textless-LINK\textgreater. This has the advantage of not
manually having to source where there is coverage, at what density and
decide on a distribution. The game revolves around being ``dropped''
into a random location on Google Street View, and having to guess where
it is located.

\textless-POTENTIALLY INSERT GEOGUESSR PICTURE\textgreater{}

Originally the player is allowed to move around, but there are modified
modes to create harder difficulties which prevent the moving or even the
panning of the camera, which is what we'll be opting for. This will also
allow it to generalize more to other static pictures than if we were
using the 360° spheres.

\textless-POTENTIALLY INSERT SAMPLE PICTURES\textgreater{}

Because different countries are of different sizes, but also have
different amounts of Google Street View coverage, deciding on a
representative distribution for generalization would be very difficult.
Instead, we opted to play the Geoguessr multiplayer game mode called
``Battle Royale: Countries'' \textless-LINK\textgreater. This game mode
revolves around trying to guess the country of a location before the
opponents do. It has a much more even distribution of countries, while
still taking into account the densities of different places.

\textless-INSERT MULTIPLAYER GRAPH\textgreater{}

Unfortunately, data collection using a multiplayer game mode is quite
slow, as even though we do not need to guess and can spectate the rest
of the game, we still need to wait for the other players to guess every
round. The number of concurrent games was also be limited by the number
of currently active players. Additionally, while spectating it is not
easily possible to get the exact coordinates of a location, restricting
us to only predicting the correct countries. Lastly, we were detected by
their anti-cheating software as the automation environment is injecting
scripts into the website.

Instead, we chose to collect data through the most popular singleplayer
game mode called ``World'' (``Classic Maps''), by putting in arbitrary
guesses and playing a lot of rounds. This allowed us to collect data a
lot quicker, as well as also collecting the coordinates, however, it
came at the cost of a very skewed distribution.

\textless-INSERT SINGLEPLAYER GRAPH\textgreater{}

To remedy this, we instead use the country distribution of our
multiplayer games and apply it to our collected singleplayer data. This
leaves a lot of data unused and forces us to remove very rare countries,
but it allows us to get the required amount of data a lot quicker.

\textless-POTENTIALLY INSERT MAPPED SINGLEPLAYER GRAPH\textgreater{}

\subsubsection{Web scraping}\label{web-scraping}

To collect this data we built our own scraper, utilizing the testing and
browser automation framework ``Playwright'' \textless-LINK\textgreater.
We then deployed 5 parallel instances of this script to a server and
periodically retrieved the newly collect data.

Our script starts off by logging and and storing the cookies for further
sessions, it then accepts the cookie conditions and attempts to start a
game. We do this by navigating the page using the text, as there are no
stable identifiers. For multiplayer it additionally checks for
rate-limiting or if it joined the same game as another instance of the
script, it those cases it waits for a certain amount of time and
attempts the same again.

After a game started it will wait for a round to start, wait for the
image to load, hide all other elements on the page and move the mouse
cursor out of the way and take a screenshot. For singleplayer it then
guesses a random location while in multiplayer it waits for the round to
end, spectating the rest of the game afterwards. At the end of each
round the coordinates or in the case of multiplayer the country are read
from the page and saved to a file. Both these files are named after the
``game id'' we extract from the URL, preventing duplicates. This is the
repeated until the script is stopped.

\textless-POTENTIALLY INSERT SCRAPING CONTROL FLOW GRAPH\textgreater{}

Initially we had a lot of issues with stability, especially with our
parallelized workers. After we got rid of hardware bottlenecks we also
looked to eliminate as many fixed waits as possible, replacing them wait
dynamic ones to avoid timing issues. Finally, we made sure to enable
auto-restarting and added a lot of other measures to completely restart
after our environment stops working, which can happen during extended
scraping sessions. We then let this script run in parallel, non-stop for
multiple weeks, collecting \textless-INSERT FIGURE\textgreater{}
multiplayer datapoints and \textless-INSERT FIGURE\textgreater{}
singleplayer datapoints.

To make sure our data is collected correctly, we manually inspected it
periodically. Any faults we noticed in the images like black screens and
blurring, we would address later in our filtering. However, we also had
to inspect whether the coordinates and countries were accurate.

(After an initial run of our singleplayer script, we noticed that the
way we collected coordinates in multiplayer did no longer work and had
been collecting incorrect coordinates for tens of thousands of images.
To address this, we built an additional script looking up the correct
coordinates using the ``game id'', this was a lot quicker than the
collection of new data, allowing us to correct the mistake quite
quickly. We also then used this new way of looking up coordinates for
our collection script.)

Also write about the ToS we searched and that is not mentioned about web
scrapping and therefore it is allowed if it is not exactly permitted
from the ToS.

\subsection{Data analysis?? (We do not have this section
now\ldots)}\label{data-analysis-we-do-not-have-this-section-now}

Introduce the part Visualize where we see the distribution?

\subsection{Data processing}\label{data-processing}

\subsubsection{Resizing of the images}\label{resizing-of-the-images}

We can't train the classifier using images in a high resolution, because
our resources are limited, and also often images (like from missing
persons) are also very low quality. So we decided to reduce the
resolution, at the beginning of the processing, about the 1/4 of the
original resolution of 1280p x 720p. This also helps to move the images
for learning to the server or also between us and also loading takes lot
less time for future processing steps.

\subsubsection{Country Enriching (Singleplayer coordinates, Multiplayer
names)}\label{country-enriching-singleplayer-coordinates-multiplayer-names}

In our project, we focused on collecting images from Geoguessr in
Singleplayer mode to ensure we obtained real images with precise
coordinates. In contrast, images collected in Multiplayer mode only
provide the actual country name without exact coordinates. Initially, we
faced a ban from Multiplayer mode due to unauthorized webpage
injections, which violated Geoguessr's policies. Additionally, we
discovered that images from Multiplayer mode often had incorrect
coordinates, leading us to concentrate our efforts on Singleplayer mode
for accurate data collection.

Our objective is to explore different approaches using low-resolution
images to evaluate their effectiveness in predicting coordinates,
regions, and countries. For this purpose, we require all three pieces of
information for each image, enabling us to train models for accurate
predictions. Throughout the process, we encountered various challenges.
Reverse geocoding allowed us to derive country information from
coordinates, but the \texttt{pycountry} module did not always provide
comprehensive location data. For example, Kosovo is not officially
recognized by \texttt{pycountry}, necessitating its manual addition to
our country list. This adjustment was crucial since Geoguessr included
information about Kosovo, which was not covered by the
\texttt{reverse\_geocoder} and \texttt{pycountry} modules.

We also faced difficulties mapping the correct country to the provided
data when names were derived from Multiplayer mode. To address this, we
implemented fuzzy matching to find the best match for each image based
on the given data from the JSON files, ensuring accurate country
assignment.

To enhance the efficiency of processing our large dataset, we employed
multiple workers to handle the data concurrently. This approach
significantly improved the processing speed by batching the dataset and
utilizing parallel processing.

(Issues with reverse geocoding, country name matching) - What exactly
issues? Did I mention all of them?

\subsubsection{Region Enriching (Source,
Mapping)}\label{region-enriching-source-mapping}

So to predict the Region of the image, we first searched for a list of
regions around the world. And decide to use the geojson file from
Natural Earth. Since for each region we had a list of coordinates, which
marks the border of the region, we had to get the middle point of each
one. Where the python library ``geopands'' comes in handy. This library
has the advantage to be able to work with geojson files and to have an
integrated middle point calculation function. In addition, we add a
unique region\_name for each region using the name of the region +
country name + id. This is needed since some region names have similar
or the same name. After this preparation, we used the middle point to
get the region for each image using their coordinates using k-nearest
neighbor method.

\subsubsection{Mapping to a
distribution}\label{mapping-to-a-distribution}

As mentioned in the previous section (Web scraping), our singleplayer
data is skewed towards a few countries, with some countries only
appearing very rarely. To address this, we are mapping our singleplayer
data to the country distribution of our multiplayer data. This allows us
to have a better distribution while still not having every country
appear with the same frequency to account for size and coverage
differences. It, however, comes with the downside of not being able to
use all of our data, although some tests showed that using all of our
data unmapped performed worse \textless-CHECK AND MENTION
RESULTS\textgreater.

Unfortunately, this also doesn't allow us to include all countries as
some of them do not appear often enough and would reduce the number of
images we are allowed to use for other countries as well. To achieve a
mapping including enough files while including as many countries as
possible, we set a minimum threshold of how often a country has to
appear within the singleplayer data (\textless-INSERT
FIGURE\textgreater). Because this included too few countries, we added a
slack factor (\textless-INSERT FIGURE\textgreater), allowing countries
that could almost meet the distribution to be included as well.

Finally, we saved this as a list of file names using our
``data-loader'', and commit it to our repository, making our runs
reproducible. We created a few different variants of the mapped list,
sometimes including more countries and other times more files per
country, until we found a good balance.

\textless-POTENTIALLY INSERT MAPPED SINGLEPLAYER GRAPH\textgreater{}

\subsubsection{Filtering of data}\label{filtering-of-data}

To address issues with our scraping's inherently unstable nature, as
well as the big variety of Google Street View images, we had to do some
automated filtering of unsuitable data. This consisted of both filtering
our images, but also the corresponding data. After filtering we again
saved this as a list of file names using our ``data-loader'', and commit
it to our repository.

To filter images we started by setting a minimum threshold of the
biggest variance of color between the pictures of an image, meaning
either red, green or blue has to vary by some amount. This easily
filters out black screens and dark images, like the ones indoor or
inside tunnels. Additionally, we added a threshold for the variance
after the laplacian kernel was applied, allowing us to filter some
blurry and low quality images. We set our thresholds after doing manual
sampling and some test runs.

\textless-INSERT SAMPLE PICTURES WITH VARIANCE\textgreater{}

Additionally, we realized that some rounds were in the exact same
locations, so we decided to filter out duplicates by comparing the
coordinates, only keeping the first image. This, as well as the image
filtering, comes with the added benefit of filtering corrupted data,
which would otherwise have to be handled in our training code.

\subsection{Prediction approaches}\label{prediction-approaches}

\subsubsection{Predicting Coordinates: Mean Squared
Error}\label{predicting-coordinates-mean-squared-error}

The initial approach we took was to predict the exact coordinates of a
location. We initially believed this would be easier since coordinate
prediction is a basic method. However, after a few training iterations,
we realized this approach was more challenging than predicting countries
or regions due to several issues.

The first problem was the distribution of our dataset compared to the
actual distribution of locations on Earth. From Geoguessr, we collected
81,505 mapped images, with a fairly even distribution after filtering,
and a total of 332,786 images without checked distribution. Despite our
efforts to ensure even distribution, the dataset was not uniformly
spread across the globe. This imbalance meant that incorrect model
predictions often resulted in large geographical errors, increasing the
loss significantly.

We observed that while the network performed well on the training set
for the first 10 epochs, the validation accuracy for distance was poor.
The model struggled to predict the correct continent in the validation
set, resulting in high mean loss. We used Mean Squared Error (MSE) and
the Haversine distance to calculate the loss, as explained in the
``Regions with Custom Loss'' subsection. Although we had a mean loss
error, it was unclear where the model was making incorrect predictions.

To address this issue, we shifted our focus to the other two approaches:
predicting countries and predicting regions. Predicting countries is a
classification problem, making it more straightforward. By predicting
regions, we can map countries to regions, allowing for a more detailed
comparison of results. This fine-grained approach helps us understand
which country and region the model believes the image was taken in,
making it easier to identify where and why the model makes incorrect
predictions and how far off these predictions are from the actual
locations.

\subsubsection{Predicting Countries: Categorical
Cross-Entropy}\label{predicting-countries-categorical-cross-entropy}

For predicting the countries, we use torch.nn.CrossEntropyLoss to
calculate the loss. This method is well-suited for our classification
problem, as it measures how well a predicted probability distribution
matches the actual distribution (or ground truth) of class labels.

First, we create a unique mapping list for each dataset, identifying
countries by their indices. This mapping allows us to convert country
names to indices and vice versa. We then map the countries from the
labels to their actual indices in all test splits. Additionally, we
calculate the loss using the raw logits from the model without applying
softmax, as CrossEntropyLoss internally handles the softmax operation.

We save this information to Weights and Biases (wandb) to ensure the
mappings are correctly and centrally stored for each run. The loss for
the predictions is calculated using the following formula:

\[
L = -\frac{1}{N} \sum_{i=1}^{N} \log \left( \frac{e^{x_{i, y_i}}}{\sum_{j} e^{x_{i,j}}} \right)
\]

where:

\begin{itemize}
\tightlist
\item
  \(L\) is the categorical cross-entropy loss.
\item
  \(N\) is the number of samples.
\item
  \(x_{i,j}\) is the predicted score/logit for class \(j\) for sample
  \(i\).
\item
  \(y_i\) is the true class label for the sample \(i\).
\end{itemize}

By implementing this method, we ensure that our model's predictions are
evaluated accurately, providing us with a reliable metric to improve our
classification models.

\subsubsection{Predicting Regions: Custom haversine smooth
loss}\label{predicting-regions-custom-haversine-smooth-loss}

For the region-prediction we use a custom loss function. Which, in short
text, is a loss function not only look if the correct region is
predicted, it considers also the distance to the correct coordinates.
Which means if the predicted region is only slightly off then the loss
is not that big like if it is far off. There is the paper ``PIGEON:
Predicting Image Geolocations'' from Stanford University, which comes in
handy for this task. They're using the haversine smooth loss function.
(Haas et al., 2024) {[}Haas et al. (2024);{]}.

\subparagraph{The steps of the custom loss
function}\label{the-steps-of-the-custom-loss-function}

The haversine distance is a measure of the shortest distance between two
points on the surface of a sphere, given their longitudes and latitudes.
It is calculated using the following formula:

\[
\text{Hav}(\mathbf{p_1}, \mathbf{p_2}) = 2r \arcsin \left( \sqrt{\sin^2 \left( \frac{\phi_2 - \phi_1}{2} \right) + \cos(\phi_1) \cos(\phi_2) \sin^2 \left( \frac{\lambda_2 - \lambda_1}{2} \right)} \right)
\]

where:

\begin{itemize}
\tightlist
\item
  \(r\) is the radius of the Earth (6371 km in this implementation),
\item
  \(\mathbf{p_1}, \mathbf{p_2}\) are the 2 points with longitude
  \(\lambda\) and latitude \(\phi\)
\end{itemize}

The smoothed labels are calculated using the following formula:

\[
y_{n,i} = \exp \left( - \frac{\text{Hav}(\mathbf{g_i}, \mathbf{x_n}) - \text{Hav}(\mathbf{g_n}, \mathbf{x_n})}{\tau} \right)
\]

where

\begin{itemize}
\tightlist
\item
  \(\mathbf{g_i}\)\hspace{0pt} are the centroid coordinates of the
  geocell polygon of cell \(\mathbf{i}\)
\item
  \(\mathbf{g_n}\)\hspace{0pt} are the centroid coordinates of the true
  geocell.
\item
  \(\mathbf{x_n}\)\hspace{0pt} are the true coordinates of the example.
\item
  \(\tau\) is a temperature parameter.
\end{itemize}

Finally, the cross-entropy loss is calculated between the model outputs
and the smoothed labels.

\begin{center}\rule{0.5\linewidth}{0.5pt}\end{center}

\textbf{Requirements:}

\begin{enumerate}
\def\labelenumi{\arabic{enumi}.}
\tightlist
\item
  \textbf{Understanding and Application:} This section allows students
  to demonstrate their understanding of various methodologies and their
  ability to apply appropriate techniques to their specific project.
\item
  \textbf{Rationale and Justification:} Discussing the methods used
  provides insight into the student's decision-making process and the
  rationale behind choosing specific approaches.
\end{enumerate}

\begin{center}\rule{0.5\linewidth}{0.5pt}\end{center}

\subsection{Model architectures}\label{model-architectures}

To develop a robust and efficient image classification model for
predicting the geographical origin of low-resolution images, we employ
several state-of-the-art convolutional neural network (CNN)
architectures. These architectures, known for their advanced design and
high performance in image recognition tasks, are utilized in our
research to ensure optimal results, as detailed in the Literature Review
above. We specifically focus on ResNet, MobileNetV2, and EfficientNet
architectures, each offering unique advantages for our project.

\textbf{Addressing the Degradation Problem in Deep Networks.} ResNet's
deep residual learning framework effectively addresses the degradation
problem in deep neural networks by allowing layers to learn residual
functions. This approach mitigates the vanishing gradient problem and
enables the training of extremely deep networks. ResNet's robustness and
generalizability are evident in its performance on the ImageNet dataset,
where the 152-layer ResNet achieved a top-5 error rate of 4.49\% (He et
al., 2015) {[}He et al. (2015);{]} . These characteristics make ResNet
an ideal baseline model for our research, aiming to predict the country
of origin for low-resolution images.

\textbf{Efficient Architectures for Mobile and Resource-Constrained
Environments.} Designed for mobile and resource-constrained
environments, MobileNetV2 represents a significant advancement in
efficient CNN architectures. The core innovation of MobileNetV2 is the
inverted residual with linear bottleneck layer module, which reduces
memory footprint and computational cost during inference without
sacrificing accuracy (Sandler et al., 2019) {[}Sandler et al.
(2019);{]}. This efficiency makes MobileNetV2 particularly suited for
our project, where we aim to develop a model that can run efficiently on
various hardware platforms while maintaining high accuracy.

\textbf{Balancing Performance and Efficiency in Model Scaling.}
EfficientNet introduces a model scaling method that uniformly scales
depth, width, and resolution using a simple yet effective compound
scaling method. This balanced scaling approach enables EfficientNet to
achieve superior performance while being much smaller and faster than
previous models (Tan and Le, 2020) {[}Tan and Le (2020);{]}.
EfficientNet's combination of high accuracy and low computational
requirements makes it an excellent choice for our project, where
hardware efficiency is critical.

By leveraging these advanced CNN architectures and fine-tuning them on
our custom dataset, we aim to develop a high-performing, efficient model
capable of accurately predicting the geographical origin of images with
minimal hardware resources.

\subsection{Training and Fine-Tuning}\label{training-and-fine-tuning}

For our project, we use the following models from selected
architectures:

\begin{itemize}
\tightlist
\item
  \textbf{ResNet}: ResNet18, ResNet34, ResNet50, ResNet101, ResNet152
\item
  \textbf{MobileNet}: MobileNetV2, MobileNetV3 Small, MobileNetV3 Large
\item
  \textbf{EfficientNet}: EfficientNet-B1, EfficientNet-B3,
  EfficientNet-B4, EfficientNet-B7
\end{itemize}

To be resource-efficient and enable training the CNN architectures with
affordable hardware, we decided to use pre-trained weights and replace
the last classification layer with a custom classification layer that
matches the classes for the countries or regions in our datasets. For
the training, we reduce the learning rate for the layers with the
pre-trained weights by a factor of 10, allowing the network to focus
more on training the new classification layer with randomly initialized
weights while fine-tuning the existing layers with pre-trained weights.
The pre-trained weights we use are the default weights from the torch
models, IMAGENET1K\_V1 and IMAGENET1K\_V2, which we loaded and used
throughout our project. Our goal is to allow the new classification
layer to learn more effectively while merely fine-tuning the existing
layers.

Additionally, we considered integrating all possible countries and
regions into the models for the final classification layer to make the
models more adaptable for multiple tasks and other datasets. This
approach could facilitate further training with more data, including
more countries and regions, at a later stage. However, we decided
against this to help the network perform better with the existing
classes in our dataset. Including too many classes would introduce
additional complexity and potential issues due to class imbalance and
insufficient representation in the training data.

\subsection{Data augmentation}\label{data-augmentation}

Data augmentation is crucial for computer vision tasks, especially when
using CNN networks. While CNNs are very powerful in identifying objects
and patterns, they struggle with variations in rotation, perspective, or
view. To address this, it is essential to augment the training data to
make the model more robust to different image transformations. This
prepares the model to recognize identical objects from different angles
or if they are stretched or squeezed. For our Geoguessr images, data
augmentation is particularly important. The images are consistently
taken by a Google car with the same height and camera settings. This
consistency helps the model learn from a uniform perspective, improving
accuracy. However, it also means the model might generalize poorly to
other datasets not taken from Google Street View.

To evaluate the impact of data augmentation, we will train our model
with and without augmentation, comparing performance on different
datasets to see how well the models generalize. This comparison will be
insightful in understanding the effectiveness of data augmentation. For
our specific case with Google Street View images, we applied the
following augmentations to the training data:

\begin{itemize}
\tightlist
\item
  \textbf{Random Resized Crop}: This augmentation randomly crops the
  image and resizes it to the original size. This helps the model learn
  to recognize objects in different parts of the image and at different
  scales, addressing the issue of objects appearing in varied locations
  within the frame.
\item
  \textbf{Random Rotation}: This augmentation randomly rotates the image
  by up to 10 degrees. It helps the model become invariant to slight
  rotations, which is important since objects in Google Street View
  images can appear slightly rotated due to changes in the car's
  movement or camera alignment.
\item
  \textbf{Color Jitter}: This augmentation randomly changes the
  brightness, contrast, saturation, and hue of the image. For example,
  varying the brightness can simulate different times of day (e.g.,
  night vs.~bright daylight), while adjusting the hue and saturation can
  account for different weather conditions or camera sensor variations.
  This makes the model more robust to lighting and color changes that
  are common in real-world scenarios.
\end{itemize}

\textgreater\textgreater\textgreater{} Add here a picture of the
training augmentation to see what it visually
\textgreater\textgreater\textgreater{}

These augmentation techniques are essential for making our model robust
and capable of generalizing to different images beyond the specific
conditions of Google Street View. By simulating various real-world
conditions, we aim to improve the model's ability to handle diverse and
unseen environments.

\subsection{Hyperparameter tuning}\label{hyperparameter-tuning}

Another method we used in this student project is hyperparameter tuning.
It is a crucial part of machine learning and helps to find the optimal
settings for the model to learn and perform at its best. During
hyperparameter tuning, all parameters were saved to Weights and Biases
(wandb), which also tracked and saved all the metrics. This platform
allowed us to organize training schedules, manage the entire training
process, and keep everything centralized. Wandb's filtering capabilities
made it easy to retrieve specific runs and compare different accuracies
for each country and region. Initially, we set some static parameters
that we did not change during hyperparameter tuning. For the optimizer,
we used the AdamW optimizer for all runs, which handles weight decay
internally. We also used a scheduler to decrease the learning rate after
a certain number of epochs to prevent overshooting the learned
parameters, with the learning rate being decreased every 10 steps.
Although these parameters could be adjusted during hyperparameter
tuning, we chose not to tune them due to time constraints and their
minimal impact on performance.

For our hyperparameter tuning, we focused on two different parameters:
learning rate and weight decay. We trained the models on five different
learning rates: 1e-1, 1e-2, 1e-3, 1e-4, and 1e-5. The learning rate
significantly impacts how well the model can learn. After initial
experiments with a broader range of learning rates, these five were the
most promising during training.

Additionally, we applied three different weight decay values: 1e-1,
1e-2, and 1e-3. Weight decay helps penalize large weights in the
network, leading to several benefits: reducing overfitting, improving
model stability, promoting feature sharing, and enhancing generalization
in over-parameterized models. These three weight decay values helped
achieve higher performance compared to not using weight decay. We did
not need L2 regularization because the AdamW optimizer handles it
internally.

\subsection{Human baseline
performance}\label{human-baseline-performance}

\subsubsection{Collection of baseline
scores}\label{collection-of-baseline-scores}

To compare our model to the performance of a human classifier, we would
first have to measure the performance of a similar human. To calculate
this, we built a small interactive application using ``Gradio''
\textless-LINK\textgreater. It loads a random image in our downscaled
resolution, though not quite as low as most of our models are trained
on, and asks the user to type in the 5 most likely countries. This then
allows us to calculate a reasonable Top-1, Top-3 and Top-5 accuracy for
comparison with our model.

Follows\ldots{}

\subsection{Machine Learning Operations
(MLOps)}\label{machine-learning-operations-mlops}

\subsubsection{Project structure}\label{project-structure}

As we did for our last project (``DSPRO1''), we are using a ``monorepo''
setup with a pipeline-style setup consisting of numbered folder and
subfolders, each representing different stages and sub-stages of our
dataflow, from data collection to model training. Every stage consists
of at least one Jupyter Notebook, with more helpers and reused python
code dispersed throughout the project. Each notebook saves the generated
data in its current folder, making the flow obvious. Within each
sub-step, the notebooks can be run in arbitrary order because they are
not inter-dependent.

\subsubsection{Handling a lot of files}\label{handling-a-lot-of-files}

Differing from our last project, however, is the amount of data. With
our scraping generating hundreds of thousands of images, we could not
store them in our git repository. Instead, we opted for storing them in
our server we had used for scraping, although in a scaled and already
enriched format, making it quicker to get our training and repository up
and running on a new machine. This server is public to allow for our
results to be reproduced.

Using a server for storage made storing the files easy, but it came with
the added challenge of reproducibility. Ideally, we would want to store
all of our data on the server but only pull the required ones for a
particular training, ensuring that they were always the same ones.

(To quickly return a list of all files present without overloading the
web server we use to serve the files, we wrote a small PHP script
returning the files names as a list of links, which can be easily
parsed.)

To solve this came up a custom set of helpers called ``data-loader''.
This would get the list of files from our server, filter them by
criteria, sort, optionally shuffle or limit them, and output the full
paths to the files that should be used for this processing step or
training. Note that each data point consists of both an image file and a
JSON file, the ``data-loader'' treats them as pairs and has stable
shuffling and limiting behavior, no matter where or how the files are
stored.

Behind the scenes, it writes a text file (``data-list'') to the
repository listing all of the files used. This file is meant to be
committed to the repository and ensures that all future runs of this
setup will get the exact same files, otherwise throw an error. If some
files were still missing locally, they are automatically downloaded
before returning the paths.

Once we had this running, we could easily deploy this on persistent
cloud environments like HSLU's GPUHub, however, we also wanted to be
able to deploy it on Google Colab \textless-LINK\textgreater. which does
not have persistent storage. To address this, we wrote a shell script
automatically clone our git repository from GitLab
\textless-LINK\textgreater, install dependencies using ``Poetry''
\textless-LINK\textgreater, convert the training notbooking to plain
python and run it.

(Even with the script, setup was still slow because hundreds of
thousands of files had to be downloaded from our server first. To solve
this, we mounted a Google Drive \textless-LINK\textgreater{} and stored
our files there. However, since the drive adapter is slow and seizes to
work with a lot of files, we had to take a couple of measures to address
this.

Firstly, we stored our downloaded files in nested directories,
containing the first and second characters in the ``game ids'' of the
files. Secondly, we store a list of all files present in the Google
Drive, preventing a slow file listing, and lastly, we store the files in
a zip file, copy the entire file and uncompress them on the local
storage of the runner. This allowed us to quickly deploy our model
training to Google Colab, which gave us the chance to rain on more
powerful GPUs.)

To speed up training in other environments, especially when using a lot
of transformations for data augmentation, we cache the prepared dataset
using pytorch right before training. The dataset is saved to a file
named after the preprocessing parameters, as well as a hash of all file
names to ensure consistency. A file only containing the test data after
the split is also saved to make calculating the metrics quicker.

For monitoring and deploying we log and push all of our run data to
``Weights and Biases'' \textless-LINK\textgreater, which allows us to
plot and compare many runs, as well as automatically do
hyperparameter-tuning. After each training we also push the model
weights as well as the test data, if it has not been saved before,
otherwise a link to it. This allows us to deploy a model and calculate
the final metrics in seconds.

To talk about:

Creating the demo for the geoguessr wizard and how we are deploying the
model in this real-world scenario

\begin{center}\rule{0.5\linewidth}{0.5pt}\end{center}

\textbf{Requirements:}

\begin{enumerate}
\def\labelenumi{\arabic{enumi}.}
\tightlist
\item
  \textbf{Core Competency in Data Science:} Data processing is a
  fundamental step in any data science project. Demonstrating this
  process shows the student's ability to handle and prepare data for
  analysis, which is a critical skill in the field.
\item
  \textbf{Transparency and Reproducibility:} Detailing the data
  processing steps ensures transparency and aids in the reproducibility
  of the results, which are key aspects of scientific research.
\end{enumerate}

\begin{center}\rule{0.5\linewidth}{0.5pt}\end{center}

\begin{center}\rule{0.5\linewidth}{0.5pt}\end{center}

\textbf{Requirements:}

\begin{enumerate}
\def\labelenumi{\arabic{enumi}.}
\tightlist
\item
  Practical Application: This section emphasizes the practical aspect of
  machine learning. It's not just about building models but also about
  deploying them effectively in real-world scenarios.
\item
  Bridging Theory and Practice: It allows students to demonstrate their
  ability to translate theoretical knowledge into practical
  applications, showcasing their readiness for industry challenges.
\end{enumerate}

\begin{center}\rule{0.5\linewidth}{0.5pt}\end{center}

\subsection{Model performance on other
datasets}\label{model-performance-on-other-datasets}

Follows\ldots{}

\begin{center}\rule{0.5\linewidth}{0.5pt}\end{center}

\textbf{Requirements:}

\begin{enumerate}
\def\labelenumi{\arabic{enumi}.}
\tightlist
\item
  Ensuring Model Reliability: Model validation is crucial for assessing
  the accuracy and reliability of the model. This section shows how the
  student evaluates the performance and generalizability of their model.
\item
  Critical Evaluation: It encourages students to critically evaluate
  their model's performance, understand its limitations, and discuss
  potential improvements.
\end{enumerate}

\begin{center}\rule{0.5\linewidth}{0.5pt}\end{center}

\section{Experiments and Results (and also
discussions)}\label{experiments-and-results-and-also-discussions}

\subsection{Predicting coordinates}\label{predicting-coordinates}

Follows\ldots{}

\[
\begin{aligned} & \text {Table 1.1. Example table for the future results :) }\\ &\begin{array}{ccc|c} \hline \hline \text { Case } & \text { Method 1 } & \text { Method 2 } & \text { Test result } \\ \hline 1 & 50 & 837 & 970 \\ 2 & 47 & 877 & 230 \\ 3 & 31 & 25 & 415 \\ 4 & 35 & 144 & 23656 \\ 5 & 45 & 300 & 556 \\ \hline
\end{array} \end{aligned}
\]

\subsection{Predicting countries}\label{predicting-countries}

Follows\ldots{}

\subsection{Predicting regions}\label{predicting-regions}

Follows\ldots{}

\subsection{General}\label{general}

Write also that we found out that a bigger data size matters more than
bigger images for all of the prediction models, which is a really nice
catch and learning.

\begin{center}\rule{0.5\linewidth}{0.5pt}\end{center}

List here all results in plots, confusion matrix and so on. The goal of
these sections is to present our result and explain how they help in
solving the problem we are working on, and to answer the research
questions we are trying to answer.

Our Hypothesis: The main goal of this student project is to determine if
an Image Classification Model can outperform humans in guessing the
countries or regions of images based solely on the image itself, without
additional information.

\begin{center}\rule{0.5\linewidth}{0.5pt}\end{center}

\section{Conclusions and Future Work}\label{conclusions-and-future-work}

Follows\ldots{}

\begin{center}\rule{0.5\linewidth}{0.5pt}\end{center}

Here we should discuss the implications of our results, our limitations,
and possible further research possibilities. We should be very honest
especially about limitations.

\begin{center}\rule{0.5\linewidth}{0.5pt}\end{center}

\section{References}\label{references}

\begin{enumerate}
\def\labelenumi{\arabic{enumi}.}
\tightlist
\item
  Suresh, Sudharshan, Nathaniel Chodosh, and Montiel Abello. ``DeepGeo:
  Photo Localization with Deep Neural Network.'' arXiv, February 16,
  2016. https://arxiv.org/abs/1810.03077.
\item
  Weyand, T., Kostrikov, I., \& Philbin, J. ``PlaNet - Photo Geolocation
  with Convolutional Neural Networks.'' arXiv, October 6, 2018.
  https://arxiv.org/abs/1602.05314.
\item
  James Hays, Alexei A. Efros. ``IM2GPS: estimating geographic
  information from a single image. Proceedings of the IEEE Conf. on
  Computer Vision and Pattern Recognition (CVPR)''. graphics, n.d.,
  2008. http://graphics.cs.cmu.edu/projects/im2gps/.
\item
  Banerjee, Arsh. ``Image Geolocation with Computer Vision,''
  arshbanerjee, May 9, 2023.
  http://www.arshbanerjee.com/uploads/paper/3bda9\_20230723185809.pdf
\item
  Haas, Lukas, Michal Skreta, Silas Alberti, and Chelsea Finn. ``PIGEON:
  Predicting Image Geolocations.'' arXiv, May 28, 2024.
  https://arxiv.org/abs/2307.05845.
\item
  Dayton, Finn, Jeffrey Heo, and Eric Werner. ``CNN Plays Geoguessr:
  Transfer Learning on ResNet50 for Classifying Street View Images,''
  cs229, n.d., 2023. https://www.finndayton.com/CS229\_Final\_Report.pdf
\item
  He, Kaiming, Xiangyu Zhang, Shaoqing Ren, and Jian Sun. ``Deep
  Residual Learning for Image Recognition.'' arXiv, December 10, 2015.
  https://arxiv.org/abs/1512.03385.
\item
  Sandler, Mark, Andrew Howard, Menglong Zhu, Andrey Zhmoginov, and
  Liang-Chieh Chen. ``MobileNetV2: Inverted Residuals and Linear
  Bottlenecks.'' arXiv, March 21, 2019.
  https://arxiv.org/abs/1801.04381.
\item
  Tan, Mingxing, and Quoc V. Le. ``EfficientNet: Rethinking Model
  Scaling for Convolutional Neural Networks.'' arXiv, September 11,
  2020. https://arxiv.org/abs/1905.11946.
\end{enumerate}

\phantomsection\label{refs}
\begin{CSLReferences}{1}{0}
\bibitem[\citeproctext]{ref-banerjee2023image}
Banerjee, Arsh. 2023. {``Image Geolocation with Computer Vision.''}
\url{http://www.arshbanerjee.com/uploads/paper/3bda9_20230723185809.pdf}.

\bibitem[\citeproctext]{ref-dayton2023cnn}
Dayton, Finn, Jeffrey Heo, and Eric Werner. 2023. {``CNN Plays
Geoguessr: Transfer Learning on ResNet50 for Classifying Street View
Images.''} \url{https://www.finndayton.com/CS229_Final_Report.pdf}.

\bibitem[\citeproctext]{ref-haas2024pigeon}
Haas, Lukas, Michal Skreta, Silas Alberti, and Chelsea Finn. 2024.
{``PIGEON: Predicting Image Geolocations.''} \emph{arXiv}.
\url{https://arxiv.org/abs/2307.05845}.

\bibitem[\citeproctext]{ref-hays2008im2gps}
Hays, James, and Alexei A. Efros. 2008. {``IM2GPS: Estimating Geographic
Information from a Single Image.''} In \emph{Proceedings of the IEEE
Conference on Computer Vision and Pattern Recognition (CVPR)}.
\url{http://graphics.cs.cmu.edu/projects/im2gps/}.

\bibitem[\citeproctext]{ref-he2015deep}
He, Kaiming, Xiangyu Zhang, Shaoqing Ren, and Jian Sun. 2015. {``Deep
Residual Learning for Image Recognition.''} \emph{arXiv}.
\url{https://arxiv.org/abs/1512.03385}.

\bibitem[\citeproctext]{ref-sandler2019mobilenetv2}
Sandler, Mark, Andrew Howard, Menglong Zhu, Andrey Zhmoginov, and
Liang-Chieh Chen. 2019. {``MobileNetV2: Inverted Residuals and Linear
Bottlenecks.''} \emph{arXiv}. \url{https://arxiv.org/abs/1801.04381}.

\bibitem[\citeproctext]{ref-suresh2016deepgeo}
Suresh, Sudharshan, Nathaniel Chodosh, and Montiel Abello. 2016.
{``DeepGeo: Photo Localization with Deep Neural Network.''}
\emph{arXiv}. \url{https://arxiv.org/abs/1810.03077}.

\bibitem[\citeproctext]{ref-tan2020efficientnet}
Tan, Mingxing, and Quoc V. Le. 2020. {``EfficientNet: Rethinking Model
Scaling for Convolutional Neural Networks.''} \emph{arXiv}.
\url{https://arxiv.org/abs/1905.11946}.

\end{CSLReferences}
